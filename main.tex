\documentclass[male, authorStatement, indexNumber, fileVersion, ]{lib/uekthesis}
\usepackage[utf8]{inputenc}

%##############################################################################
% Global variables !!! - ważne by kodowanie UTF-8 było wstawione przed nimi
%
% To zmiennne do modyfikacji przez użytkownika
%
\globalFullAuthor{Maciej Sypień}    % Pełna nazwa autora pracy
\globalShortAuthor{M. Sypień}       % Autor - zwięzła forma wydruku
\globalFullTitle{Przygotowanie pracy dyplomowej \\[2mm] wraz z systemem LaTeX}  % Pełny tytuł pracy
\globalShortTitle{Praca dyplomowa w systemie \LaTeXe}     % Krótki, zwięzły tytuł pracy
\globalFullUniversity{Uniwersytet Ekonomiczny w Krakowie} % Pełna nazwa uniwersytetu
\globalShortUniversity{UEK}                           % Skrócona nazwa uniwersytetu
\globalDepartment{Wydział Zarządzania}                % Wydział
\globalDegreeprogramme{Informatyka Stosowana}         % Kierunek studiów
\globalThesisType{Praca dyplomowa}                    % Typ pracy dyplomowej
\globalSupervisor{prof. n. dr hab. Jana Iksińskiego}  % Promotor
\globalAcknowledgements{Dla moich rodziców oraz najbliższych przyjaciół za niezłomną wiarę w~moje zwycięstwo.}   % Podziękowania
\globalFileVersion{0.1.0}   % wersja pliku
\globalIndexNumber{123456}  % wersja pliku
\globalCity{Kraków}         % miasto
\globalYear{2014}           % rok powstania pracy
\globalKeywords{nauka, komputery, praca dyplomowa, latex, uczelnia, student} % słowa kluczowe dla pracy

%##############################################################################
%
\setlength{\cftsecnumwidth}{10mm}

%##############################################################################
% Zmienia przypisy na ciągłe w całym dokumencie - używa do tego pakietu "chngcntr"
\counterwithout{footnote}{chapter}

%##############################################################################
% Ustawienia dla bibliografii
\DeclareQuoteAlias{croatian}{polish}
\usepackage[
  language=polish,  % Ustawienie języka
  %urldate=long,     % Wyświetlenie długiego formatu w datach
  style=numeric,    % numeric, alphabetic, authoryear,
  sorting=nty,      % Sortowanie: nAME, tITLE, yEAR
  isbn=true,        % zgodnie z nowym rozporządzeniem  pole obowiązkowe
  firstinits=true,  % will render all first and middle names as initials.
  abbreviate=true,
  autolang=hyphen,
  backend=biber,]{biblatex}
\addbibresource{bibliography.bib}  % dołączenie pliku bibliografii

%##############################################################################
% pakiet dla podpisów np: dla lstlintings
\usepackage[
  font=small,
  format=hang,
  labelformat=simple,
  justification=justified,
  singlelinecheck=false,
  labelsep=colon
]{caption}

\usepackage{lipsum}

%##############################################################################
% Dodanie komendy dla caption + source
\newcommand*{\captionsource}[2]{%
  \caption[{#1}]{%
    #1%
    \\\hspace{\linewidth}%
    {Źródło:} #2%
  }%
}

%##############################################################################
% zmiana nazw opisu dla tabel i rysunków i innych elementów
\addto\captionspolish{%
    \renewcommand{\figurename}{Rysunek}%
    %\renewcommand{\listfigurename}{Spis rysuneczków}%  nazwa testowa ;)
    \renewcommand{\tablename}{Tabela}%
    \renewcommand{\listtablename}{Spis tabel}%
    % \renewcommand\lstlistingname{Kod źródłowy}
    \renewcommand\lstlistlistingname{Spis kodów źródłowych}
}

%##############################################################################
% Zdefiniowanie nowego, dodatkowego jezyka
\lstdefinelanguage{Biblatex}{
    basicstyle=\ttfamily,
    keywords={@Manual, @Book, @Article, @Book, @Online, },%
    keywordstyle=\color{blue}\bfseries,%
    ndkeywords={title, subtitle, author,year, month, day, version, language, url, number, volume, pages, journaltitle, isbn, address, publisher},%
    ndkeywordstyle=\color{Emerald}\bfseries,%
    %identifierstyle=\color{yellow},%
    sensitive=false,%
    commentstyle=\color{Gray}\ttfamily,%
    comment=[l]{\%},%
    stringstyle=\color{BrickRed}\ttfamily,%
    morestring=[b]",%
    morestring=[s]{\{}{\}},%
    morestring=[s][\color{black}]{\{Kun}{,},% dla strongów od '{Kun' do ','
    morestring=[s][\color{black}]{\{Popper84}{,},% dla strongów od '{Popp:' do ','
    morestring=[s][\color{black}]{\{man:}{,},% dla strongów od '{man:' do ','
    morestring=[s][\color{black}]{\{url:}{,}% dla strongów od '{url:' do ','
}

%##############################################################################
% Zdefiniowanie nowego stylu LaTeX2e
\lstdefinestyle{LaTeX2e}{
  language=[LaTeX]TeX,
  keywordstyle=\color{Red},
  basicstyle=\ttfamily,
  alsoletter={\#},
  morekeywords={chapter},
  morestring=[b][\color{ForestGreen}]\$,
  morestring=[b][\color{ForestGreen}]\$\$
}

%##############################################################################
% Zdefiniowanie nowego stylu Bash - poprawiającego wyświetlanie niektórych komend
\lstdefinestyle{Bash}{
  language=Bash,
  keywordstyle=\color{MidnightBlue},
  basicstyle=\ttfamily,
  alsoletter={-},
  sensitive=false,
  morekeywords={sudo, wget, chmod, dpkg, git},
  %literate={\$}{{\textcolor{MidnightBlue}{\$}}}1
  %         {-}{{\textcolor{MidnightBlue}{-}}}1
  %         {~}{{\textcolor{MidnightBlue}{\textasciitilde}}}1,
}

%##############################################################################
% Ustawienie opcji dla wyświetlania kodów źródłowych
% Wiecej: pakiet listings s. 26
\lstset{
    %extendedchars=true, % celowo wyłączone
    inputencoding=utf8,
    basicstyle=\footnotesize\ttfamily,
    numberstyle=\footnotesize,
    captionpos=t,
    tabsize=2,
    frame=tb,
    numbers=left,
    breaklines=true,
    showtabs=false,
    showspaces=false,
    showstringspaces=false,
    %language=[LaTeX]TeX,   % domyślny język wyłączony
    style=LaTeX2e,          % domyślny styl dla listingów = LaTeX2e
    keywordstyle=\color{MidnightBlue},
    commentstyle=\color{OliveGreen},
    stringstyle=\color{Red},
    emph={label},
    literate={ą}{{\k{a}}}1
             {ć}{{\'c}}1
             {ę}{{\k{e}}}1
             {ó}{{\'o}}1
             {ń}{{\'n}}1
             {ł}{{\l{}}}1
             {ś}{{\'s}}1
             {ź}{{\'z}}1
             {ż}{{\.z}}1
             {Ą}{{\k{A}}}1
             {Ć}{{\'C}}1
             {Ę}{{\k{E}}}1
             {Ó}{{\'O}}1
             {Ń}{{\'N}}1
             {Ł}{{\L{}}}1
             {Ś}{{\'S}}1
             {Ź}{{\'Z}}1
             {Ż}{{\.Z}}1,
}

% jednolity wygląd jak pozostałych spisów - tabel, rysunków, bibliografii
\makeatletter
\begingroup\let\newcounter\@gobble\let\setcounter\@gobbletwo
  \globaldefs\@ne \let\c@loldepth\@ne
  \newlistof{listings}{lol}{\lstlistlistingname}
\endgroup
\let\l@lstlisting\l@listings
\let\my@chapter\@chapter
\renewcommand*{\@chapter}{%
  \addtocontents{lol}{\protect\addvspace{10pt}}%
  \my@chapter}
\makeatother
\renewcommand{\lstlistoflistings}{\listoflistings}

% odstępy dla poszczególnych wierszy
\makeatletter
% 1.5em - odstęp od początku lewej strony tekstu
% 2em - odstęp po numerze odwołującym sie do elementu (np: 4.3)
\def\l@lstlisting#1#2{\@dottedtocline{1}{1.5em}{2.3em}{#1}{#2}}
\makeatother


%##############################################################################
% Zdefiniowanie własnych nazw dla poszczególnych definicji, twierdzeń itp
\theoremstyle{plain}
\newtheorem{thm}{Twierdzenie}[section]
\newtheorem{lem}[thm]{Lemat}      % w matematyce twierdzenie pomocnicze
\newtheorem{prop}[thm]{Założenie}
\newtheorem*{cor}{Wniosek}

\theoremstyle{definition}
\newtheorem{defn}[thm]{Definicja}
\newtheorem{conj}{Przypuszczenie}[section]
\newtheorem{exmp}{Przykład}[section]

\theoremstyle{remark}
\newtheorem*{rem}{Spostrzeżenie}
\newtheorem*{note}{Notka}

%##############################################################################
% Defninicja komendy zmieniająca początkowy duży znak litery na mały w odnośniku do tego rozdziału, sekcji, podsekcji, ect.

% Lowercase \nameref = \lnameref
\newcommand{\lnameref}[1]{%
\bgroup
\let\nmu\MakeLowercase
\nameref{#1}\egroup}

% First UpperCase then lowercase = \fnameref
\newcommand{\fnameref}[1]{%
\bgroup
\def\nmu{\let\nmu\MakeLowercase}%
\nameref{#1}\egroup}

% helper for lowercase newcommands
\newcommand{\nmu}{}

%##############################################################################
% Ustawienia dla pakietu biblatex
\renewcommand*{\newunitpunct}{\addcomma\space}
%\renewbibmacro*{in:}{}
\renewbibmacro{in:}{}

\xpatchbibdriver{book}{%
\newunit
\iffieldundef{maintitle}
{\printfield{volume}%
\printfield{part}}
{}%
}
{%
}{}{}

\xpatchbibdriver{book}{%
\usebibmacro{publisher+location+date}%
}
{%
\usebibmacro{publisher+location+date}%
\newunit
\printfield{volume}%
\printfield{part}
\usebibmacro{finentry}
}{}{}

\DeclareFieldFormat*{volume}{tom\addnbspace #1}
\DeclareFieldFormat{journaltitle}{\mkbibquote{#1}}
\DeclareFieldFormat[article,periodical]{number}{nr~#1}% number of a journal

\DeclareFieldFormat
[article,inbook,incollection,inproceedings,patent,thesis,unpublished]
{title}{\mkbibemph{#1}}
%
\renewbibmacro*{volume+number+eid}{%
  \printfield{volume}%
  \setunit{\addcomma\space}%
  \printfield{number}%
  \setunit{\addcomma\space}%
  \printfield{eid}}

\renewbibmacro*{journal+issuetitle}{%
  \usebibmacro{journal}%
  \setunit*{\addspace}%
  \iffieldundef{series}
  {}
  {
    \newunit
    \printfield{series}%
    \setunit{\addspace}}%
    \usebibmacro{issue+date}%
    \setunit{\addspace}%
    \usebibmacro{issue}%
    \setunit{\addspace}%
    \usebibmacro{volume+number+eid}%
    \newunit
}

\renewbibmacro*{issue+date}{%
  \iffieldundef{issue}
  {\usebibmacro{date}}
  {
     \printfield{issue}%
     \setunit*{\addspace}%
     \usebibmacro{date}
  }%
  \newunit
}

\renewbibmacro*{publisher+location+date}{%
  \printlist{publisher}%
  \iflistundef{publisher}
  {\setunit*{\addcomma\space}}
  {\setunit*{\addcomma\space}}%
  \printlist{location}%
  \setunit*{\addspace}%
  \usebibmacro{date}%
  \newunit
}

%##############################################################################
% Definicje własnych, nowych makr - makra wymagają pakietu 'xcolor'
\newcommand{\red}[1]{{\color{RedOrange}{#1}}}
\newcommand{\yellow}[1]{{\color{Dandelion}{#1}}}
\newcommand{\green}[1]{{\color{LimeGreen}{#1}}}
\newcommand{\RED}[1]{{\colorbox{RedOrange}{#1}}}
\newcommand{\YELLOW}[1]{{\colorbox{Dandelion}{#1}}}
\newcommand{\GREEN}[1]{{\colorbox{LimeGreen}{#1}}}



%##############################################################################
% Koniec preambuły i rozpoczęcie treści właściwej dokumentu
\begin{document}
\nocite{*}

\titlepages
\tableofcontents
\clearpage

\chapter{Wstępu słów kilka}
\label{cha:wstepu_slow_kilka}


%------------------------------------------------------------------------------
\section{Tekst}
\label{sec:tekst}
\LaTeX{} zapewnia autorowi doskonałe narzędzie w~zarządzaniu tworzonym dokumentem z wyszczególnieniem numerowania sekcji, wypunktowań oraz odwołań do tabel, rysunków, wzorów oraz~wielu innych elementów. \\
Przykładowo w~bardzo prosty sposób możemy odwołać się do wzoru \ref{eq:rownowaznosc_masy_i_energii} umieszczonego w~kolejnym podrozdziale zatytułowanym \nameref{sec:matematyka}.

Pisząc dokument nie trzeba się również martwić o wygląd dokumentu --- \LaTeX{} wyręczy w~tym pisarza --- natomiast autor może skupić całą swoją uwagę na właściwej treści pracy, bez obawy, że jakiś styl zmieni się przez przypadek podczas pracy z dokumentem jak często ma to miejsce w edytorach WYSIWYG.


%------------------------------------------------------------------------------
\section{Matematyka}
\label{sec:matematyka}
Wzory prezentowane poniżej ukazują możliwości \LaTeX{} w~zakresie składu formuł matematycznych. Wzory są numerowane dynamicznie według kolejności, podobnie jak pozostałe elementy o~których mowa w~sekcji~\ref{sec:tekst}.

\begin{equation}
    \label{eq:rownowaznosc_masy_i_energii}
    E = mc^2
\end{equation}

, gdzie

\begin{equation}
    m = \frac{m_0}{\sqrt{1-\frac{v^2}{c^2}}}.
\end{equation}

Warto wspomnieć też, że jest możliwość wypisania wzoru w jednej linijce w sposób następujący $f(x) = ax^2 + bx + c$, gdzie $a \ne 0$, lub posłużyć się tą samą notacją tylko z zastosowaniem podwójnego znaku dolara przez co uzyskamy efekt $$f(x) = ax^2 + bx + c$$, gdzie $a \ne 0$. \\

Jak widać przy wzorach pisanych z zastosowaniem pojedynczego i podwójnego znaku dolara nie ma z boku strony numeru referencyjnego do wzoru tworzonego dynamicznie przez \LaTeX{}a, natomiast ulega zmianie układ formatowania.

%------------------------------------------------------------------------------
\section{Tabele}
\label{sec:tabele}
Środowisko \LaTeX{}a zapewnia doskonałe wsparcie do przedstawiania różnego rodzaju informacji w postaci tabel. Począwszy od stosunkowo małych i prostych (np.: tabela~\ref{tab:wyrownanie_wartosci_liczbowych}) poprzez te, które swoją objętością mogą rozciągać się na wiele stron (tabela \ref{tab:multiline_table}).

\begin{table}[h]
\caption{Wyrównanie wartości liczbowych w tabelach}
\label{tab:wyrownanie_wartosci_liczbowych}
\centering
    \begin{tabular}{|l|c|r|r@{.}l|}
    \hline
    Do lewej strony & Do środka strony & Do prawej strony & \multicolumn{2}{c|}{Do kropki} \\
    \hline
    \hline
    123.4  & 123.4   & 123.4   & 123 & 4\\
    12.34  & 12.34   & 12.34   & 12 & 34\\
    1.234  & 1.234   & 1.234   & 1 & 234\\
    0.1234 & 0.1234  & 0.1234  & 0 & 1234\\
    \hline
    \end{tabular}
\caption*{Źródło: Opracowanie własne na podstawie \parencite{man:Szpyrka13}.}
\end{table}


\begin{center}
\begin{longtable}{|l|l|l|}
\caption{Wielowierszowa tabela}
\label{tab:multiline_table} \\

\hline \multicolumn{1}{|c|}{\textbf{Time (s)}} & \multicolumn{1}{c|}{\textbf{Triple chosen}} & \multicolumn{1}{c|}{\textbf{Other feasible triples}} \\ \hline
\endfirsthead

\multicolumn{3}{c}%
{{\tablename\ \thetable{} -- kontynuacja z poprzedniej strony}} \\
\hline \multicolumn{1}{|c|}{\textbf{Time (s)}} &
\multicolumn{1}{c|}{\textbf{Triple chosen}} &
\multicolumn{1}{c|}{\textbf{Other feasible triples}} \\ \hline
\endhead

\hline \multicolumn{3}{|r|}{{Kontynuacja na następnej stronie}} \\ \hline
\endfoot

\hline \hline
\endlastfoot

0 & (1, 11, 13725) & (1, 12, 10980), (1, 13, 8235), (2, 2, 0), (3, 1, 0) \\
2745 & (1, 12, 10980) & (1, 13, 8235), (2, 2, 0), (2, 3, 0), (3, 1, 0) \\
5490 & (1, 12, 13725) & (2, 2, 2745), (2, 3, 0), (3, 1, 0) \\
8235 & (1, 12, 16470) & (1, 13, 13725), (2, 2, 2745), (2, 3, 0), (3, 1, 0) \\
10980 & (1, 12, 16470) & (1, 13, 13725), (2, 2, 2745), (2, 3, 0), (3, 1, 0) \\
13725 & (1, 12, 16470) & (1, 13, 13725), (2, 2, 2745), (2, 3, 0), (3, 1, 0) \\
16470 & (1, 13, 16470) & (2, 2, 2745), (2, 3, 0), (3, 1, 0) \\
19215 & (1, 12, 16470) & (1, 13, 13725), (2, 2, 2745), (2, 3, 0), (3, 1, 0) \\
21960 & (1, 12, 16470) & (1, 13, 13725), (2, 2, 2745), (2, 3, 0), (3, 1, 0) \\
24705 & (1, 12, 16470) & (1, 13, 13725), (2, 2, 2745), (2, 3, 0), (3, 1, 0) \\
27450 & (1, 12, 16470) & (1, 13, 13725), (2, 2, 2745), (2, 3, 0), (3, 1, 0) \\
30195 & (2, 2, 2745) & (2, 3, 0), (3, 1, 0) \\
32940 & (1, 13, 16470) & (2, 2, 2745), (2, 3, 0), (3, 1, 0) \\
35685 & (1, 13, 13725) & (2, 2, 2745), (2, 3, 0), (3, 1, 0) \\
38430 & (1, 13, 10980) & (2, 2, 2745), (2, 3, 0), (3, 1, 0) \\
41175 & (1, 12, 13725) & (1, 13, 10980), (2, 2, 2745), (2, 3, 0), (3, 1, 0) \\
43920 & (1, 13, 10980) & (2, 2, 2745), (2, 3, 0), (3, 1, 0) \\
46665 & (2, 2, 2745) & (2, 3, 0), (3, 1, 0) \\
49410 & (2, 2, 2745) & (2, 3, 0), (3, 1, 0) \\
52155 & (1, 12, 16470) & (1, 13, 13725), (2, 2, 2745), (2, 3, 0), (3, 1, 0) \\
54900 & (1, 13, 13725) & (2, 2, 2745), (2, 3, 0), (3, 1, 0) \\
57645 & (1, 13, 13725) & (2, 2, 2745), (2, 3, 0), (3, 1, 0) \\
60390 & (1, 12, 13725) & (2, 2, 2745), (2, 3, 0), (3, 1, 0) \\
63135 & (1, 13, 16470) & (2, 2, 2745), (2, 3, 0), (3, 1, 0) \\
65880 & (1, 13, 16470) & (2, 2, 2745), (2, 3, 0), (3, 1, 0) \\
68625 & (2, 2, 2745) & (2, 3, 0), (3, 1, 0) \\
71370 & (1, 13, 13725) & (2, 2, 2745), (2, 3, 0), (3, 1, 0) \\
74115 & (1, 12, 13725) & (2, 2, 2745), (2, 3, 0), (3, 1, 0) \\
76860 & (1, 13, 13725) & (2, 2, 2745), (2, 3, 0), (3, 1, 0) \\
79605 & (1, 13, 13725) & (2, 2, 2745), (2, 3, 0), (3, 1, 0) \\
82350 & (1, 12, 13725) & (2, 2, 2745), (2, 3, 0), (3, 1, 0) \\
85095 & (1, 12, 13725) & (1, 13, 10980), (2, 2, 2745), (2, 3, 0), (3, 1, 0) \\
87840 & (1, 13, 16470) & (2, 2, 2745), (2, 3, 0), (3, 1, 0) \\
90585 & (1, 13, 16470) & (2, 2, 2745), (2, 3, 0), (3, 1, 0) \\
93330 & (1, 13, 13725) & (2, 2, 2745), (2, 3, 0), (3, 1, 0) \\
96075 & (1, 13, 16470) & (2, 2, 2745), (2, 3, 0), (3, 1, 0) \\
98820 & (1, 13, 16470) & (2, 2, 2745), (2, 3, 0), (3, 1, 0) \\
101565 & (1, 13, 13725) & (2, 2, 2745), (2, 3, 0), (3, 1, 0) \\
104310 & (1, 13, 16470) & (2, 2, 2745), (2, 3, 0), (3, 1, 0) \\
107055 & (1, 13, 13725) & (2, 2, 2745), (2, 3, 0), (3, 1, 0) \\
109800 & (1, 13, 13725) & (2, 2, 2745), (2, 3, 0), (3, 1, 0) \\
112545 & (1, 12, 16470) & (1, 13, 13725), (2, 2, 2745), (2, 3, 0), (3, 1, 0) \\
115290 & (1, 13, 16470) & (2, 2, 2745), (2, 3, 0), (3, 1, 0) \\
118035 & (1, 13, 13725) & (2, 2, 2745), (2, 3, 0), (3, 1, 0) \\
120780 & (1, 13, 16470) & (2, 2, 2745), (2, 3, 0), (3, 1, 0) \\
123525 & (1, 13, 13725) & (2, 2, 2745), (2, 3, 0), (3, 1, 0) \\
126270 & (1, 12, 16470) & (1, 13, 13725), (2, 2, 2745), (2, 3, 0), (3, 1, 0) \\
129015 & (2, 2, 2745) & (2, 3, 0), (3, 1, 0) \\
131760 & (2, 2, 2745) & (2, 3, 0), (3, 1, 0) \\
134505 & (1, 13, 16470) & (2, 2, 2745), (2, 3, 0), (3, 1, 0) \\
137250 & (1, 13, 13725) & (2, 2, 2745), (2, 3, 0), (3, 1, 0) \\
139995 & (2, 2, 2745) & (2, 3, 0), (3, 1, 0) \\
142740 & (2, 2, 2745) & (2, 3, 0), (3, 1, 0) \\
145485 & (1, 12, 16470) & (1, 13, 13725), (2, 2, 2745), (2, 3, 0), (3, 1, 0) \\
148230 & (2, 2, 2745) & (2, 3, 0), (3, 1, 0) \\
150975 & (1, 13, 16470) & (2, 2, 2745), (2, 3, 0), (3, 1, 0) \\
153720 & (1, 12, 13725) & (2, 2, 2745), (2, 3, 0), (3, 1, 0) \\
156465 & (1, 13, 13725) & (2, 2, 2745), (2, 3, 0), (3, 1, 0) \\
159210 & (1, 13, 13725) & (2, 2, 2745), (2, 3, 0), (3, 1, 0) \\
161955 & (1, 13, 16470) & (2, 2, 2745), (2, 3, 0), (3, 1, 0) \\
164700 & (1, 13, 13725) & (2, 2, 2745), (2, 3, 0), (3, 1, 0) \\
\end{longtable}
\end{center}



%------------------------------------------------------------------------------
\section{Rysunki}
\label{sec:rysunki}
\LaTeX umożliwia także dołączanie do dokumentu fotografii wraz opisem, odwołaniem oraz źródłem tak samo jak w innych obiektach wymienionych w rozdziale \ref{sec:tekst}.



\chapter{Merytoryczna strona pracy}

% pierwsza sekcja
\section{Pierwsze symptomy}
\label{sec:pierwsze_symptomy}
\lipsum[2-4]

\chapter{Praktyczna strona pracy}
\label{cha:rozdzialTrzeci}

Rozdział zawierający praktyczną cześć pracy dyplomanta. Jego charakter nie musi być odkrywczy, lecz powinien ukazywać wkład pracy autora w~rozwój omawianego zagadnienia.

% pierwsza sekcja
\section{Podpunkt pierwszy}
\label{sec:podpunkt_pierwszy}
\lipsum[34-35]

% druga sekcja
\section{Podpunkt drugi}
\label{sec:podpunkt_drugi}
\lipsum[100-102]

% Załączniki
% \appendix
% \include{dodatekA}
% \include{dodatekB}
% itd.

%##############################################################################
% Poniżej odnajdziesz spisy dołączane do dokumentu (w tym do spisu treści).
% Jesli nie chcesz wykorszystać któregoś z nich w pracy, zakomentuj go

% Spis tabel
\clearpage
\cleardoublepage
\phantomsection
\addcontentsline{toc}{chapter}{\listtablename}
\listoftables

% Spis rysunków
\clearpage
\cleardoublepage
\phantomsection
\addcontentsline{toc}{chapter}{\listfigurename}
\listoffigures

% Spis kodów źródłowych
\cleardoublepage
\phantomsection
\addcontentsline{toc}{chapter}{\lstlistlistingname}
\lstlistoflistings

% Bibliografia
\clearpage
\printbibliography[heading=bibintoc]

\end{document}
