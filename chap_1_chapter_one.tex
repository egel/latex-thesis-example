\chapter{Wstępu słów kilka}
\label{cha:wstepu_slow_kilka}


%------------------------------------------------------------------------------
\section{Tekst}
\label{sec:tekst}
\LaTeX{} zapewnia autorowi doskonałe narzędzie w~zarządzaniu tworzonym dokumentem z wyszczególnieniem numerowania sekcji, wypunktowań oraz odwołań do tabel, rysunków, wzorów oraz~wielu innych elementów. \\
Przykładowo w~bardzo prosty sposób możemy odwołać się do wzoru \ref{eq:rownowaznosc_masy_i_energii} umieszczonego w~kolejnym podrozdziale zatytułowanym \nameref{sec:matematyka}.

Pisząc dokument nie trzeba się również martwić o wygląd dokumentu --- \LaTeX{} wyręczy w~tym pisarza --- natomiast autor może skupić całą swoją uwagę na właściwej treści pracy, bez obawy, że jakiś styl zmieni się przez przypadek podczas pracy z dokumentem jak często ma to miejsce w edytorach WYSIWYG.


%------------------------------------------------------------------------------
\section{Matematyka}
\label{sec:matematyka}
Wzory prezentowane poniżej ukazują możliwości \LaTeX{} w~zakresie składu formuł matematycznych. Wzory są numerowane dynamicznie według kolejności, podobnie jak pozostałe elementy o~których mowa w~sekcji~\ref{sec:tekst}.

\begin{equation}
    \label{eq:rownowaznosc_masy_i_energii}
    E = mc^2
\end{equation}

, gdzie

\begin{equation}
    m = \frac{m_0}{\sqrt{1-\frac{v^2}{c^2}}}.
\end{equation}

Warto wspomnieć też, że jest możliwość wypisania wzoru w jednej linijce w sposób następujący $f(x) = ax^2 + bx + c$, gdzie $a \ne 0$, lub posłużyć się tą samą notacją tylko z zastosowaniem podwójnego znaku dolara przez co uzyskamy efekt $$f(x) = ax^2 + bx + c$$, gdzie $a \ne 0$. \\

Jak widać przy wzorach pisanych z zastosowaniem pojedynczego i podwójnego znaku dolara nie ma z boku strony numeru referencyjnego do wzoru tworzonego dynamicznie przez \LaTeX{}a, natomiast ulega zmianie układ formatowania.

%------------------------------------------------------------------------------
\section{Tabele}
\label{sec:tabele}
Środowisko \LaTeX{}a zapewnia doskonałe wsparcie do przedstawiania różnego rodzaju informacji w postaci tabel. Począwszy od stosunkowo małych i prostych (tabela~\ref{tab:wyrownanie_wartosci_liczbowych} oraz~\ref{tab:odmiana_slowa_latex}) poprzez te, które swoją objętością mogą rozciągać się na wiele stron (tabela \ref{tab:odmiana_slowa_latex}).

\begin{table}[h]
\caption{Wyrównanie wartości liczbowych w tabelach}
\label{tab:wyrownanie_wartosci_liczbowych}
\centering
    \begin{tabular}{|l|c|r|r@{.}l|}
    \hline
    Do lewej strony & Do środka strony & Do prawej strony & \multicolumn{2}{c|}{Do kropki} \\
    \hline
    \hline
    123.4  & 123.4   & 123.4   & 123 & 4\\
    12.34  & 12.34   & 12.34   & 12 & 34\\
    1.234  & 1.234   & 1.234   & 1 & 234\\
    0.1234 & 0.1234  & 0.1234  & 0 & 1234\\
    \hline
    \end{tabular}
\caption*{Źródło: Opracowanie własne na podstawie, M. Marcin Szpyrka, LaTeX Podstawy użytkowania. Przygotowanie pracy dyplomowej.}
\end{table}


\begin{table}[h]
\caption{Odmiana słowa \LaTeX ~przez przypadki.}
\label{tab:odmiana_slowa_latex}
\centering
	\begin{tabular}{| l | c | c | c |} \hline
	\textbf{Przypadek} & \textbf{Pytanie} & \textbf{Żywotny} & \textbf{Nieżywotny} \\ \hline \hline
	Mianownik    & Kto? Co?        & \multicolumn{2}{c|}{ \tt{LaTeX} } \\ \hline
	Dopełniacz   & Kogo? Czego?    & \tt{LaTeXa}      & \tt{LaTeXu} \\ \hline
	Celownik     & Komu? Czemu?    & \multicolumn{2}{c|}{ \tt{LaTeXowi} } \\ \hline
	Biernik      & Kogo? Co?       & \tt{LaTeXa}      & \tt{LaTeX} \\ \hline
	Narzędnik    & Z kim? Z czym?  & \multicolumn{2}{c|}{ \tt{LaTeXem} } \\ \hline
	Miejscownik  & O kim? O czym?  & \multicolumn{2}{c|}{ \tt{LaTeXu} } \\ \hline
	Wołacz       & O!              & \multicolumn{2}{c|}{ \tt{LaTeXu} } \\ \hline
	\end{tabular}
\end{table}


\begin{center}
\begin{longtable}{|l|l|l|}
\caption{Wielowierszowa tabela}
\label{tab:multiline_table} \\

\hline \multicolumn{1}{|c|}{\textbf{Time (s)}} & \multicolumn{1}{c|}{\textbf{Triple chosen}} & \multicolumn{1}{c|}{\textbf{Other feasible triples}} \\ \hline
\endfirsthead

\multicolumn{3}{c}%
{{\bfseries \tablename\ \thetable{} -- kontynuacja z poprzedniej strony}} \\
\hline \multicolumn{1}{|c|}{\textbf{Time (s)}} &
\multicolumn{1}{c|}{\textbf{Triple chosen}} &
\multicolumn{1}{c|}{\textbf{Other feasible triples}} \\ \hline
\endhead

\hline \multicolumn{3}{|r|}{{Kontynuacja na następnej stronie}} \\ \hline
\endfoot

\hline \hline
\endlastfoot

0 & (1, 11, 13725) & (1, 12, 10980), (1, 13, 8235), (2, 2, 0), (3, 1, 0) \\
2745 & (1, 12, 10980) & (1, 13, 8235), (2, 2, 0), (2, 3, 0), (3, 1, 0) \\
5490 & (1, 12, 13725) & (2, 2, 2745), (2, 3, 0), (3, 1, 0) \\
8235 & (1, 12, 16470) & (1, 13, 13725), (2, 2, 2745), (2, 3, 0), (3, 1, 0) \\
10980 & (1, 12, 16470) & (1, 13, 13725), (2, 2, 2745), (2, 3, 0), (3, 1, 0) \\
13725 & (1, 12, 16470) & (1, 13, 13725), (2, 2, 2745), (2, 3, 0), (3, 1, 0) \\
16470 & (1, 13, 16470) & (2, 2, 2745), (2, 3, 0), (3, 1, 0) \\
19215 & (1, 12, 16470) & (1, 13, 13725), (2, 2, 2745), (2, 3, 0), (3, 1, 0) \\
21960 & (1, 12, 16470) & (1, 13, 13725), (2, 2, 2745), (2, 3, 0), (3, 1, 0) \\
24705 & (1, 12, 16470) & (1, 13, 13725), (2, 2, 2745), (2, 3, 0), (3, 1, 0) \\
27450 & (1, 12, 16470) & (1, 13, 13725), (2, 2, 2745), (2, 3, 0), (3, 1, 0) \\
30195 & (2, 2, 2745) & (2, 3, 0), (3, 1, 0) \\
32940 & (1, 13, 16470) & (2, 2, 2745), (2, 3, 0), (3, 1, 0) \\
35685 & (1, 13, 13725) & (2, 2, 2745), (2, 3, 0), (3, 1, 0) \\
38430 & (1, 13, 10980) & (2, 2, 2745), (2, 3, 0), (3, 1, 0) \\
41175 & (1, 12, 13725) & (1, 13, 10980), (2, 2, 2745), (2, 3, 0), (3, 1, 0) \\
43920 & (1, 13, 10980) & (2, 2, 2745), (2, 3, 0), (3, 1, 0) \\
46665 & (2, 2, 2745) & (2, 3, 0), (3, 1, 0) \\
49410 & (2, 2, 2745) & (2, 3, 0), (3, 1, 0) \\
52155 & (1, 12, 16470) & (1, 13, 13725), (2, 2, 2745), (2, 3, 0), (3, 1, 0) \\
54900 & (1, 13, 13725) & (2, 2, 2745), (2, 3, 0), (3, 1, 0) \\
57645 & (1, 13, 13725) & (2, 2, 2745), (2, 3, 0), (3, 1, 0) \\
60390 & (1, 12, 13725) & (2, 2, 2745), (2, 3, 0), (3, 1, 0) \\
63135 & (1, 13, 16470) & (2, 2, 2745), (2, 3, 0), (3, 1, 0) \\
65880 & (1, 13, 16470) & (2, 2, 2745), (2, 3, 0), (3, 1, 0) \\
68625 & (2, 2, 2745) & (2, 3, 0), (3, 1, 0) \\
71370 & (1, 13, 13725) & (2, 2, 2745), (2, 3, 0), (3, 1, 0) \\
74115 & (1, 12, 13725) & (2, 2, 2745), (2, 3, 0), (3, 1, 0) \\
76860 & (1, 13, 13725) & (2, 2, 2745), (2, 3, 0), (3, 1, 0) \\
79605 & (1, 13, 13725) & (2, 2, 2745), (2, 3, 0), (3, 1, 0) \\
82350 & (1, 12, 13725) & (2, 2, 2745), (2, 3, 0), (3, 1, 0) \\
85095 & (1, 12, 13725) & (1, 13, 10980), (2, 2, 2745), (2, 3, 0), (3, 1, 0) \\
87840 & (1, 13, 16470) & (2, 2, 2745), (2, 3, 0), (3, 1, 0) \\
90585 & (1, 13, 16470) & (2, 2, 2745), (2, 3, 0), (3, 1, 0) \\
93330 & (1, 13, 13725) & (2, 2, 2745), (2, 3, 0), (3, 1, 0) \\
96075 & (1, 13, 16470) & (2, 2, 2745), (2, 3, 0), (3, 1, 0) \\
98820 & (1, 13, 16470) & (2, 2, 2745), (2, 3, 0), (3, 1, 0) \\
101565 & (1, 13, 13725) & (2, 2, 2745), (2, 3, 0), (3, 1, 0) \\
104310 & (1, 13, 16470) & (2, 2, 2745), (2, 3, 0), (3, 1, 0) \\
107055 & (1, 13, 13725) & (2, 2, 2745), (2, 3, 0), (3, 1, 0) \\
109800 & (1, 13, 13725) & (2, 2, 2745), (2, 3, 0), (3, 1, 0) \\
112545 & (1, 12, 16470) & (1, 13, 13725), (2, 2, 2745), (2, 3, 0), (3, 1, 0) \\
115290 & (1, 13, 16470) & (2, 2, 2745), (2, 3, 0), (3, 1, 0) \\
118035 & (1, 13, 13725) & (2, 2, 2745), (2, 3, 0), (3, 1, 0) \\
120780 & (1, 13, 16470) & (2, 2, 2745), (2, 3, 0), (3, 1, 0) \\
123525 & (1, 13, 13725) & (2, 2, 2745), (2, 3, 0), (3, 1, 0) \\
126270 & (1, 12, 16470) & (1, 13, 13725), (2, 2, 2745), (2, 3, 0), (3, 1, 0) \\
129015 & (2, 2, 2745) & (2, 3, 0), (3, 1, 0) \\
131760 & (2, 2, 2745) & (2, 3, 0), (3, 1, 0) \\
134505 & (1, 13, 16470) & (2, 2, 2745), (2, 3, 0), (3, 1, 0) \\
137250 & (1, 13, 13725) & (2, 2, 2745), (2, 3, 0), (3, 1, 0) \\
139995 & (2, 2, 2745) & (2, 3, 0), (3, 1, 0) \\
142740 & (2, 2, 2745) & (2, 3, 0), (3, 1, 0) \\
145485 & (1, 12, 16470) & (1, 13, 13725), (2, 2, 2745), (2, 3, 0), (3, 1, 0) \\
148230 & (2, 2, 2745) & (2, 3, 0), (3, 1, 0) \\
150975 & (1, 13, 16470) & (2, 2, 2745), (2, 3, 0), (3, 1, 0) \\
153720 & (1, 12, 13725) & (2, 2, 2745), (2, 3, 0), (3, 1, 0) \\
156465 & (1, 13, 13725) & (2, 2, 2745), (2, 3, 0), (3, 1, 0) \\
159210 & (1, 13, 13725) & (2, 2, 2745), (2, 3, 0), (3, 1, 0) \\
161955 & (1, 13, 16470) & (2, 2, 2745), (2, 3, 0), (3, 1, 0) \\
164700 & (1, 13, 13725) & (2, 2, 2745), (2, 3, 0), (3, 1, 0) \\
\end{longtable}
\end{center}



%------------------------------------------------------------------------------
\section{Rysunki}
\label{sec:rysunki}
\LaTeX umożliwia także dołączanie do dokumentu fotografii wraz opisem, odowłaniem oraz źródłem tak samo jak w innych pozycjach wymienionych w rozdziale \ref{sec:tekst}.


