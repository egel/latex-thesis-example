\documentclass[12pt]{lib/uekthesis}
\usepackage{polski}
\usepackage[utf8]{inputenc}

% zdefiniowane kolory
\definecolor{wine}{RGB}{149,56,56}
\definecolor{wine-stain}{RGB}{200,125,125}
\definecolor{grass-green}{RGB}{53,161,103}
\definecolor{ocean-blue}{RGB}{9,63,97}

% dodatkowe pakiety
\usepackage{enumerate}     %
\usepackage{multirow}      % pakiet dla tabel wielowierszowych
\usepackage{caption}       % pakiet dla podpisów np: dla lstlintings
\usepackage{listings}      % pakiet dla kodów zródłowych
\usepackage{lipsum}        % pakiet wyświetlający "Lorem impsum..."
\usepackage{longtable}     % pakiet wielostronicowych tabel


\lstloadlanguages{TeX}
\lstset{
  literate={ą}{{\k{a}}}1
           {ć}{{\'c}}1
           {ę}{{\k{e}}}1
           {ó}{{\'o}}1
           {ń}{{\'n}}1
           {ł}{{\l{}}}1
           {ś}{{\'s}}1
           {ź}{{\'z}}1
           {ż}{{\.z}}1
           {Ą}{{\k{A}}}1
           {Ć}{{\'C}}1
           {Ę}{{\k{E}}}1
           {Ó}{{\'O}}1
           {Ń}{{\'N}}1
           {Ł}{{\L{}}}1
           {Ś}{{\'S}}1
           {Ź}{{\'Z}}1
           {Ż}{{\.Z}}1,
  language=bash,
  captionpos=b,
  tabsize=3,
  frame=lines,
  keywordstyle=\color{blue},
  commentstyle=\color{darkgreen},
  stringstyle=\color{red},
  numbers=left,
  numberstyle=\textnormal,
  numbersep=12pt,
  breaklines=true,
  showstringspaces=false,
  basicstyle=\ttfamily,
  emph={label}
}

%---------------------------------------------------------------------------

\theoremstyle{plain}
\newtheorem{thm}{Theorem}[section]
\newtheorem{lem}[thm]{Lemma}
\newtheorem{prop}[thm]{Proposition}
\newtheorem*{cor}{Corollary}

\theoremstyle{definition}
\newtheorem{defn}[thm]{Definition}
\newtheorem{conj}{Conjecture}[section]
\newtheorem{exmp}{Przykład}[section]

\theoremstyle{remark}
\newtheorem*{rem}{Remark}
\newtheorem*{note}{Note}

%------------------------------Zaawansowane--------------------------------
\author{Maciej Sypień}
\shortAuthor{M. Sypień}

\titlePL{Jak przygotować wspaniałą pracę dyplomową \\[2mm] wraz z~systemem \LaTeXe}
\titleEN{How to perpare a beautiful thisis in \LaTeXe}

\shortTitlePL{Praca dyplomowa w systemie \LaTeXe} % skrócona wersja tytułu jeśli jest bardzo długi
\shortTitleEN{Preparing thesis in \LaTeXe}

\thesisType{Praca dyplomowa}
%\thesistype{Master of Science Thesis}

\supervisor{dr hab. Jan Iksiński, prof. n.}
%\supervisor{Jan Kowalski PhD, DSc}

\degreeprogramme{Informatyka Stosowana}
%\degreeprogramme{Applied Computer Science}

\date{2014}

%\department{Katedra Informatyki Stosowanej}
%\department{Department of Applied Computer Science}

\department{Wydział Zarządzania}
%\faculty{Wydział Elektrotechniki, Automatyki,\protect\\[-1mm] Informatyki i Inżynierii Biomedycznej}
%\faculty{Faculty of Electrical Engineering, Automatics, Computer Science and Biomedical Engineering}

\acknowledgements{Dla moich rodziców oraz najbliższych przyjaciół za niezłomną wiarę w~moje zwycięstwo.}

\setlength{\cftsecnumwidth}{10mm}

%---------------------------------------------------------------------------

\hypersetup{
  %bookmarks=true,         % pokaż/ukryj pasek zakładek w trakcie wyświetlania dokumentu Acrobatem
  unicode=true,           % używać znaków z alfabetów niełacińskich zakładkach Acrobata
  pdftoolbar=true,        % show Acrobat’s toolbar?
  pdfmenubar=true,        % show Acrobat’s menu?
  pdffitwindow=false,     % window fit to page when opened
  pdfstartview={FitH},    % fits the width of the page to the window
  pdftitle={},            % title
  pdfauthor={Maciuś},     % author
  pdfsubject={Subject},   % subject of the document
  pdfcreator={Creator},   % creator of the document
  pdfproducer={Producer}, % producer of the document
  pdfkeywords={keyword1} {key2} {key3}, % list of keywords
  pdfnewwindow=true,      % links in new window
  linktoc=page,           % Ustawienie linków dla bibliografi (none, all, page, section)
  colorlinks=true,        % false: boxed links; true: colored links
  linkcolor=wine,         % color of internal links (change box color with linkbordercolor)
  citecolor=grass-green,  % color of links to bibliography
  filecolor=magenta,      % color of file links
  urlcolor=ocean-blue,    % color of external links
}

%---------------------------------------------------------------------------
% Definicje własnych, nowych makr
%---------------------------------------------------------------------------

% poniższe makra wymagają pakietu 'xcolor'
\newcommand{\red}[1]{ {\color{RedOrange}{#1}} }
\newcommand{\yellow}[1]{ {\color{Dandelion}{#1}} }
\newcommand{\green}[1]{ {\color{LimeGreen}{#1}} }
\newcommand{\RED}[1]{ {\colorbox{RedOrange}{#1}} }
\newcommand{\YELLOW}[1]{ {\colorbox{Dandelion}{#1}} }
\newcommand{\GREEN}[1]{ {\colorbox{LimeGreen}{#1}} }


%---------------------------------------------------------------------------

\begin{document}
\bibliographystyle{abbrv}

\titlepages

% ---------------------- Streszczenie ----------------------
\begin{abstract}
Dokument prezentuje podstawowe zasady składu tekstu w~systemie \LaTeX{} z~wyszczególnieniem zastosowania nomenklatury języka do tworzenia rozdziałów, sekcji oraz podsekcji charakterystycznych dla prac dyplomowych. \\
Zestawiono w nim wyłącznie praktyczne przykłady użycia różnego rodzaju obiektów, w~tym tabel, rysunków, wzorów matematycznych wraz z~,,dobrymi praktykami'' ułatwiającymi stworzenie estetycznych dokumentów oraz późniejszą efektywną pracę z nimi. \\
\end{abstract}

\tableofcontents
\clearpage

\chapter{Wstępu słów kilka}

% pierwsza sekcja
\section{Tekst}
\label{sec:tekst}
\LaTeX\ zapewnia autorowi tekstu pomoc w~zarządzaniu numerowaniem sekcji, wypunktowaniami oraz odwołaniami do tabel, rysunków i~innych elementów. \\
Przykładowo w~bardzo prosty sposób możemy odwołać się do wzoru \ref{eqn:wzor1} umieszczonego w kolejnym rodziale.

% druga sekcja
\section{Matematyka}
\label{sec:matematyka}
Poniższy wzór prezentuje możliwości \LaTeX\ w~zakresie składu formuł matematycznych. Wzory są numerowane automatycznie, podobnie jak inne elementy o~których mowa w~sekcji~\ref{sec:tekst}.

\begin{equation}
    E = mc^2,
    \label{eqn:wzor1}
\end{equation}

gdzie

\begin{equation}
    m = \frac{m_0}{\sqrt{1-\frac{v^2}{c^2}}}.
\end{equation}
\chapter{Merytoryczna strona pracy}

% pierwsza sekcja
\section{Tekst}
\label{sec:tekst}
\lipsum[2-4]

\chapter{Praktyczna strona pracy}
\label{cha:rozdzialTrzeci}

Rozdział zawierający praktyczną cześć pracy dyplomanta. Jego charakter nie musi być odkrywczy, lecz powinien ukazywać wkład pracy dyplomanta w~rozwój omawianego zagadnienia.

% pierwsza sekcja
\section{Podpunkt pierwszy}
\label{sec:podpunkt_pierwszy}
\lipsum[34-35]

% druga sekcja
\section{Podpunkt drugi}
\label{sec:podpunkt_drugi}
\lipsum[100-102]





% itd.
% \appendix
% \include{dodatekA}
% \include{dodatekB}
% itd.


\bibliography{bibliografia}
%\begin{thebibliography}{1}
%
%\bibitem{Dil00}
%A.~Diller.
%\newblock {\em LaTeX wiersz po wierszu}.
%\newblock Wydawnictwo Helion, Gliwice, 2000.
%
%\bibitem{Lam92}
%L.~Lamport.
%\newblock {\em LaTeX system przygotowywania dokumentów}.
%\newblock Wydawnictwo Ariel, Krakow, 1992.
%
%\bibitem{Alvis2011}
%M.~Szpyrka.
%\newblock {\em {On Line Alvis Manual}}.
%\newblock AGH University of Science and Technology, 2011.cccccc
%\newblock \\\texttt{http://fm.ia.agh.edu.pl/alvis:manual}.
%
%\end{thebibliography}

\end{document}
